\documentclass[11pt]{scrreprt}
\usepackage[sexy]{evan}
\newcommand{\HH}{\mathbb H}

\begin{document}
\title{Notes on 0xPARC Stuff}

\maketitle

\chapter{PCP}
\section{Overview}
Imagine that Penny has a long proof of some mathematical theorem, and Victor wants to verify it.
\emph{A priori}, this would require Victor to actually read every line of the proof.
As anyone who has experience grading students knows, that's a whole lot of work.
A lazy grader might try to save time by only ``spot checking''
a small number of lines of the proof.
However, it would be really easy to cheat such a grader:
all you have to do is make a single wrong step somewhere
and hope that particular step is not one of the ones examined by Victor.

A probabilistically checkable proof is a way to try to circumvent this issue.
It provides a protocol where Penny can format her proof such that
Victor can get high-probability confidence of its confidence
by only checking $K$ bits of the proof at random, for some absolute constant $K$.
This result is one part of the \emph{PCP theorem} which is super famous.

The purpose of this post is to give a high level summary of the ideas that go into the
protocol and convince you that the result is possible, because at first glance it seems
absurd that a single universal constant $K$ is going to be enough.
The post is divided into several sections.
\begin{itemize}
  \ii Explanation of the general \emph{sum-check protocol},
  which is super powerful and used
  \href{https://zkproof.org/2020/03/16/sum-checkprotocol/}{in many other contexts}.
  (Quote: ``when designing an efficient interactive proof system,
  there is only one hammer you need to have in your toolbox:
  the sum-check protocol of Lund, Fortnow, Karloff, and Nisan.'')

  \ii A description of low-degree testing, which lets you take a
  giant table of values and verify whether it's a multivariable polynomial.

  \ii Statement of the Quad-SAT problem,
  which is the NP-complete problem which we'll be using for this post:
  an instance of Quad-SAT is a bunch of quadratic equations in multiple variables
  that one wants to find a solution to.

  \ii Staple together everything above to get a toy PCP protocol.

  \ii An improvement using error-correcting codes.
\end{itemize}
The first three sections commute.

As this is a big picture overview, we will generally not prove things;
detailed proofs are usually spelled out in textbooks,
and I no longer have any desire to continue writing more textbooks than I already have.

\section{The general sum-check protocol}
Before worrying about systems of equations, let's imagine we just have a
\emph{single} equation
\[ Z_1 + Z_2 + \dots + Z_{\text{big}} = H \]
for some variables $Z_i$ and constant $H \in \FF_q$, all over $\FF_q$,
and assume further that $q$ is not too small.

Imagine a prover Penny has a value assigned to each $Z_i$,
and is asserting to the verifier Victor they $Z_i$'s sum to $H$.
Victor wants to know that Penny computed the sum $H$ correctly,
but Victor doesn't want to actually read all the values of $Z_i$.

Well, at face value, this is an obviously impossible task.
Even if Victor knew all but one of Penny's $Z_i$'s, that wouldn't be good enough.

So to get anywhere, we need to to give Victor at least one magic power.

\subsection{An oracle to a multilinear polynomial}
Assume for convenience that the number of $Z$'s happens to be $2^n$
and change notation to a function $f \colon \{0,1\}^n \to \FF_q$, so our equation becomes
\[ \sum_{\vec v \in \{0,1\}^n} f(\vec v) = H. \]
In other words, we have changed notation so that our variables are indexed over a
hypercube: from $f(0, \dots, 0)$ to $f(1, \dots, 1)$.

Now here's the magic power we're granting.
By polynomial interpolation, no matter what function $f$ we had initially,
we can view it as multilinear polynomial $P \in \FF_q[X_1, \dots, X_n]$.
For example, suppose $n=3$ and the eight (arbitrary) variable values were given
\begin{align*}
  f(0,0,0) &= 8 \\
  f(0,0,1) &= 15 \\
  f(0,1,0) &= 8 \\
  f(0,1,1) &= 15 \\
  f(1,0,0) &= 8 \\
  f(1,0,1) &= 15 \\
  f(1,1,0) &= 17 \\
  f(1,1,1) &= 29.
\end{align*}
(So $H = 8+15+8+15+8+15+17+29 = 115$.)
Then we'd be trying to fill in the blanks in the equation
\[ P(x,y,z) = {\square} + {\square} x + {\square} y + {\square} z
  + {\square} xy + {\square} yz + {\square} zx + {\square} xyz \]
so that $P$ agrees with $f$ on the cube.
This comes down to solving a system of linear equations;
in this case it turns out that $P(x,y,z) = 5xyz + 9xy + 7z + 8$ works,
and I've cherry-picked the numbers so a lot of the coefficients work out to $0$ for
convenience, but math majors should be able to verify that $P$ exists and is unique
no matter what eight initial numbers I would have picked (by induction on $n$).

Now here's the magic power:
we are going to let Victor make \emph{one} call to a magic oracle
that can tell Victor the value of $P(r_1,\dots,r_n)$,
for his choice of $(r_1, \dots, r_n) \in \FF_q^n$.
Note importantly that the $r_i$'s do not have to $0$/$1$,
in fact we will say Victor just chooses them randomly from the much larger $\FF_q$.
But he can only ask the oracle for that single value of $P$,
and otherwise has no idea what any of the $Z_i$'s are.
The punch line of the protocol is that this single oracle call is good enough.
If Victor has this oracle, he only needs to read one value for
Penny to convince him that $H$ was computed correctly.

\subsection{A playthrough of the sum-check protocol}
Let's use the example above with $n=3$:
Penny has chosen those eight values with $H = 115$,
and wants to convince Victor without actually sending all eight values.
Penny has done her homework and computed the coefficients of $P$ as well
(after all, she chose the values of $f$), so Penny can evaluate $P$ anywhere she wants.
But Victor can only ask the oracle about a single value of the polynomial $P$
on a point (probably) outside the hypercube.

Here's how they do it.
(All the information sent by Penny to Victor is boxed.)

\begin{enumerate}
  \ii Penny announces her claim $\boxed{H = 115}$.

  \ii They now discuss the first coordinate:
  \begin{itemize}
    \ii Victor asks Penny to evaluate the linear one-variable polynomial
    \[ g_1(T) \coloneqq P(T,0,0) + P(T,0,1) + P(T,1,0) + P(T,1,1) \]
    and send the result. In our example, it equals
    \[ g_1(T) = 8 + 15 + (9T+8) + (14T+15) = \boxed{23T+46}. \]

    \ii Victor then checks that this $g_1$ is consistent with the claim $H=115$;
    it should satisfy $H = g_1(0) + g_1(1)$ by definition.
    Indeed, $g_1(0)+g_1(1) = 46+69 = 115 = H$.

    \ii Finally, Victor commits to a random choice of $r_1 \in \FF_q$; let's say $r_1 = 7$.
    From now on, he'll always use $7$ for the first argument to $P$.
  \end{itemize}

  \ii With the first coordinate fixed at $r_1 = 7$, they talk about the second coordinate:
  \begin{itemize}
    \ii Victor asks Penny to evaluate the linear polynomial
    \[ g_2(U) \coloneqq P(7,U,0) + P(7,U,1). \]
    and send the result. In our example, it equals
    \[ g_2(U) = (63U+8) + (98U+15) = \boxed{161U + 23}. \]

    \ii Victor makes sure the claimed $g_2$ is consistent with $g_1$;
    it should satisfy $g_1(r_1) = g_2(0)+g_2(1)$.
    Indeed, it does $g_1(7) = 23 \cdot 7 + 46 = 23 + 184 = g_2(0) + g_2(1)$.

    \ii Finally, Victor commits to a random choice of $r_2 \in \FF_q$; let's say $r_1 = 3$.
    From now on, he'll always use $3$ for the second argument to $P$.
  \end{itemize}

  \ii They now settle the last coordinate:
  \begin{itemize}
    \ii Victor asks Penny to evaluate the linear polynomial
    \[ g_3(U) \coloneqq P(7,3,V) \]
    and send the result. In our example, it equals
    \[ g_3(U) = \boxed{112V+197}. \]

    \ii Victor makes sure the claimed $g_3$ is consistent with $g_2$;
    it should satisfy $g_2(r_2) = g_3(0)+g_3(1)$.
    Indeed, it does $g_2(3) = 161 \cdot 3 + 23 = 197 + 309 = g_3(0) + g_3(1)$.

    \ii Finally, Victor commits to a random choice of $r_3 \in \FF_q$; let's say $r_3 = -1$.
  \end{itemize}

  \ii Victor has picked all three coordinates, and is ready consults the oracle.
  He gets $P(7,3,-1) = 85$.
  This matches $g_3(-1) = 85$, and the protocol ends.
\end{enumerate}

\subsection{General procedure}
The previous transcript should generalize obviously to any $n > 3$,
but we spell it out anyways.
Penny has already announced $H$ and pre-computed $P$.
Now for $i = 1, \dots, n$,
\begin{itemize}
\ii Victor asks Penny to compute the univariate polynomial $g_i$
corresponding to partial sum, where the $i$th parameter is a free parameter
while all the $r_1$, \dots, $r_{i-1}$ have been fixed already.
\ii Victor sanity-checks each of Penny's answer by making sure $g_i$ is consistent
with (that is, $g_{i-1}(r_{i-1}) = g_i(0) + g_i(1)$,
or for the edge case $i=1$ that $H = g_1(0) + g_1(1)$).
\ii Then Victor commits to a random $r_i \in \FF_q$ and moves on to the next coordinate.
\end{itemize}
Once Victor has decided on every $r_i$, he asks the oracle for $P(r_1, \dots, r_n)$
and makes sure that it matches the value of $g_n(r_n)$.
If so, Victor believes Penny.

Up until now, we wrote the sum-check protocol as a sum over $\{0,1\}^n$.
However, actually there is nothing in particular special about $\{0,1\}^n$
and it would work equally well with $\HH^n$ for any small finite set $\HH$;
the only change is that the polynomial $P$
would now have degree at most $|\HH|-1$ in each variable,
rather than being multilinear.
Accordingly, the $g_i$'s change from being linear to up to degree $|\HH|-1$.
Everything else stays the same.

\section{Low-degree testing}
\subsection{The issue}
In the general sum-check protocol we just described,
we have what looks like a pretty neat procedure,
but the elephant in the room is that we needed to make a call to some oracle.
Just one call, which we denoted $P(r_1, \dots, r_n)$, but you need an oracle nonetheless.
Also, the protocol we just described was interactive;
but it would be nice to make it non-interactive.

When we do PCP, the eventual plan
we are going to replace both the oracle call and Penny's answers
with a printed phone book.
The phone book contains, among other things, a big table mapping every $n$-tuple
$(r_1, \dots, r_n) \in \FF_q^n$ to its value $P(r_1, \dots, r_n)$
that Penny mails to Victor via Fedex or whatever.
This is of course a lot of work for Penny to print and mail this phone book.
However, Victor doesn't care about Penny's shipping costs.
After all, it's not like he's \emph{reading} the phone book;
he's just checking the entries he needs.
(That's sort of the point of a phone book, right?)

But now there's a new issue.
How can we trust the entries of the phone book are legitimate?
After all, Penny is the one putting it together.
If Penny was trying to fool Victor, she could write whatever she wanted
(``hey Victor, the value of $P$ is $42$ at every input'')
and then just lie during the sum-check protocol.
After all, she knows Victor isn't actually going to read the whole phone book.

\subsection{The goal}
The answer to this turns out be \emph{low-degree testing}.
For example, in the case where $|\HH| = 2$ we described earlier,
there is a promise that $P$ is supposed a multilinear polynomial.
For example, this means that
\[ \frac58 P(100, r_2, \dots, r_n) + \frac38 P(900, r_2, \dots, r_n) = P(400, r_2, \dots, r_n) \]
should be true.
If Victor randomly samples equations like this, and they all check out,
he can be confident that $P$ is probably ``mostly'' a polynomial.

I say ``mostly'' because, well, there's no way to verify the whole phone book.
By definition, Victor is trying to avoid reading.
Imagine if Penny makes a typo somewhere in the phone book ---
well, there's no way to notice it, because that entry will never see daylight.
However, Victor \emph{also} doesn't care about occasional typos in the phone book.
For his purposes, he just wants to check the phone book is 99\% accurate,
since the sum-check protocol only needs to read a few entries anyhow.

\subsection{The procedure --- the line-versus-point test}
This is now a self-contained math problem, so I'll just write the statement
and not prove it (the proof is quite difficult).
Suppose $g \colon \FF_q^m \to \FF_q$ is a function
and we want to see whether or not it's a polynomial of total degree at most $d$.

The procedure goes as follows.
The prover prints out two additional posters containing large tables $B_0$ and $B_1$,
whose contents are defined as follows:
\begin{itemize}
  \ii In the table $B_0$, for each point $\vec b \in \FF_q^m$,
  the prover writes $g(\vec b) \in \FF_q$.
  %   {For experts:
  %   If you're reading really carefully,
  %   you might notice that this is technically redundant with the phone book.
  %   The pedagogical cartoon I'm trying to draw is to make the phone book
  %   consist of all the stuff needed for sum-check,
  %   and the poster consist of all the stuff needed for low-degree testing.
  %   If you want to make the tiny optimization of only sending the $B_2$ poster,
  %   go ahead, I guess.}
  We denote the entry of the table by $B_0[\vec b]$.

  \ii In the table $B_1$, for each \emph{line}
  \[ \left\{ \vec b + t \vec m \mid t \in \FF_q \right\}. \]
  the prover writes the restriction of $g$ to the line $\ell$,
  which is a single-variable polynomial of degree at most $d$ in $\FF_q[t]$.
  We denote the entry of the table by $B_1[\ell]$.
\end{itemize}
The verifier then does the obvious thing:
pick a random point $\vec b$, a random line $\ell$ through it,
and check the tables $B_0$ and $B_1$ are consistent.

This simple test turns out to be good enough, though proving this is hard
and requires a lot of math.
But the statement of the theorem is simple:
\begin{theorem}
  [The line-versus-point test]
  There are absolute constants $C, c > 0$ such that:

  Let $d \ge 1$ be an integer.
  Suppose the tables $B_0$ and $B_1$ pass the line-versus-point test
  with probability at least $\eps \ge \frac{d^C m^C}{q^c}$.
  Then there in fact does exist a polynomial $P \in \FF_q^m \to \FF_q$
  of degree at most $d$, such that
  \begin{itemize}
    \ii At least $\Omega(\eps)$ fraction of the entries of the table $B_0$
    satisfy $B_0[\vec b] = P(\vec b)$
    \ii At least $\Omega(\eps)$ fraction of the entries of the table $B_1$
    satisfy $B_1[\ell] = P|_\ell$.
  \end{itemize}
\end{theorem}

\section{Quad-SAT}
As all NP-complete problems are equivalent, we can pick any one which is convenient.
Systems of linear equations don't make for good NP-complete problems,
but quadratic equations do.
So we are going to use Quad-SAT,
in which one has a bunch of variables over a finite field $\FF_q$,
a bunch of equations in these variables of degree at most two,
and one wishes to find a satisfying assignment.

\begin{remark*}
  [QSAT is pretty obviously NP-complete]
  If you can't see right away that QSAT is NP-complete,
  the following example instance can help,
  showing how to convert any instance of 3-SAT into a QSAT problem:
  \begin{align*}
    x_i^2 &= x_i \qquad \text{ for each } i=1,2,\dots,10000 \\
    y_1 &= (1-x_{42}) \cdot x_{17}, \hspace{3.5em} 0 = y_1 \cdot x_{53} \\
    y_2 &= (1-x_{19}) \cdot (1-x_{52}), \hspace{1em} 0 = y_2 \cdot (1-x_{75}) \\
    y_3 &= x_{25} \cdot x_{64}, \hspace{6em} 0 = y_3 \cdot x_{81} \\
    &\vdotswithin= \qquad\text{(imagine many more such pairs of equations)}
  \end{align*}
  The $x_i$'s are variables which are seen to either be $0$ or $1$.
  And then each pair of equations with $y_i$ corresponds to a clause of 3-SAT.
\end{remark*}

Let's say there are $N$ variables and $E$ equations, and $N$ and $E$ are both large.
Penny has worked really hard and figured out a satisfying assignment
\[ A \colon \{x_1, \dots, x_N\} \to \FF_q \]
and wants to convince Victor she has this $A$.
Victor really hates reading,
so Victor neither wants to read all $N$ values of the $x_i$
nor plug them into each of the $E$ equations.
He's fine receiving lots of stuff in the mail; he just doesn't want to read it.

\section{A toy PCP protocol for Quad-SAT}
We now have enough tools to describe a quad-SAT protocol that will break
the hearts of Fedex drivers everywhere.
In summary, the overview of this protocol is going to be the following:
\begin{itemize}
  \ii Penny prints $q^E$ phone books, one phone book each for each linear combination
  of the given Q-SAT equations.
  We'll describe the details of the phone book contents later.

  \ii Penny additionally prints the two posters corresponding
  to a low-degree polynomial extension of $A$
  (we describe this exactly in the next section).

  \ii Victor picks a random phone book and runs sum-check on it.

  \ii Victor runs a low-degree test on the posters.

  \ii Victor makes sure that the phone book value he read is consistent with the posters.
\end{itemize}
Let's dive in.

\subsection{Setup}
In sum-check, we saw we needed a bijection of $[N]$ into $\HH^m$.
So let's fix this notation now (it is annoying, I'm sorry).
We'll let $\HH$ be a set of size $|\HH| \coloneqq \log (N)$
and set $m = \log_{|\HH|} N$.
This means we have a bijection from $\{1, \dots, N\} \to \HH^m$,
so we can rewrite the type-signature of $A$ to be
\[ A \colon \HH^m \to \FF_q. \]

The contents of the phone books will take us a while to describe,
but we can actually describe the posters right now, and we'll do so.
Earlier when describing sum-check, we alluded to the following theorem,
but we'll state it explicitly now:
\begin{theorem}
  [Existence of the low-degree extension]
  Suppose $\phi \colon \HH^n \to \FF_q$ is \emph{any} function.
  Then there exists a unique polynomial $\wt\phi \colon \FF_q^n \to \FF_q$,
  which agrees with $\phi$ on the values of $\HH^n$
  and has degree at most $|\HH|+1$ in each coordinate.
  Moreover, this polynomial $\wt\phi$ can be easily computed given the values of $\phi$.
\end{theorem}
\begin{proof}
  Lagrange interpolation and induction on $m$.
\end{proof}
We saw this earlier in the special case $\HH=\{0,1\}$ and $n=3$,
where we constructed the multilinear polynomial $5xyz+9xy+7z+8$ out of
eight initial values.

In any case, the posters are generated as follows.
Penny takes her known assignment $A \colon \HH^m \to \FF_q$
and extends it to a polynomial
\[ \wt A \in \FF_q[T_1, \dots, T_m] \]
using the above theorem;
by abuse of notation, we'll also write $\wt A \colon \FF_q^m \to \FF_q$.
She then prints the two posters we described earlier for the point-versus-line test.

\subsection{Taking a random linear combination}
The first step of the reduction is to try and generate just a single equation to check,
rather than have to check all of them.
There is a straightforward (but inefficient; we'll improve it later) way to do this:
take a \emph{random} linear combination of the equations
(there are $q^E$ possible combinations).

To be really verbose, if $\mathcal{E}_1$, \dots, $\mathcal{E}_E$ were the equations,
Victor picks random weights $\lambda_1$, \dots, $\lambda_E$ in $\FF_q$
and takes the equation $\lambda_1 \mathcal{E}_1 + \dots + \lambda_E \mathcal{E}_E$.
In fact, imagine the title on the cover of the phone book is
given by the weights $(\lambda_1, \dots, \lambda_E) \in \FF_q^m$.
Since both parties know $\mathcal E_1$, \dots, $\mathcal E_E$,
they agree on which equation is referenced by the weights.

We'll just check \emph{one} such random linear combination.
This is good enough because, in fact,
if an assignment of the variables fails even one of the $E$ equations,
it will fail the collated equation with probability $1 - 1/q$ --- exactly!
(To see this, suppose that equation $\mathcal E_1$ was failed by the assignment $A$.
Then, for any fixed choice of $\lambda_2$, \dots, $\lambda_E$, there is always
exactly one choice of $\lambda_1$ which makes the collated equation true,
while the other $q-1$ all fail.)

To emphasize again: Penny is printing $q^E$ phone books right now and we only use one.
Look, I'm sorry, okay?

\subsection{Sum-checking the equation (or: how to print the phone book)}
Let's zoom in on one linear combination to use sum-check on.
(In other words, pick only one of the phone books at random.)
Let's agree to describe the equation using the notation
\[
  c = \sum_{\vec\imath \in \HH^m} \sum_{\vec\jmath \in \HH^m}
  a_{\vec\imath, \vec\jmath} x_{\vec\imath} \cdot x_{\vec\jmath}
  + \sum_{\vec\imath \in \HH^m} b_{\vec\imath} x_{\vec\imath}.
\]
In other words, we've changed notation so both the variables
and the coefficients are indexed by vectors in $\HH^m$.
When we actually implement this protocol, the coefficients need to be actually computed:
they came out of $\lambda_1 \mathcal{E}_1 + \dots + \lambda_E {\mathcal E}_E$.
(So for example, the value of $c$ above is given
by $\lambda_1$ times the constant term of $\mathcal E_1$,
plus $\lambda_2$ times the constant term of $\mathcal E_2$, etc.)

Our sum-check protocol that we talked about earlier
used a sequence $(r_1, \dots, r_n) \in \{0,1\}^n$.
For our purposes, we have these quadratic equations,
and so it'll be convenient for us if we alter the protocol to use pairs
$(\vec\imath, \vec\jmath) \in \FF_q^m \times \FF_q^m$ instead.
In other words, rather than $f(\vec v)$
our variables will be indexed instead in the following way:
\begin{align*}
  f &\colon \HH^m \times \HH^m \to \FF_q \\
  f(\vec\imath, \vec\jmath) &\coloneqq
    a_{\vec\imath, \vec\jmath} A(\vec\imath) A(\vec\jmath)
    + \frac{1}{|\HH|^m} b_{\vec\imath} A(\vec\imath).
\end{align*}
Hence Penny is trying to convince Victor that
\[ \sum_{\vec\imath \in \FF_q^m}
  \sum_{\vec\jmath \in \FF_q^m} f(\vec\imath, \vec\jmath) = c. \]

In this modified sum-check protocol, Victor picks the indices two at a time.
So in the step where Victor picked $r_1$ in the previous step,
he instead picks $i_1$ and $j_1$ at once.
Then instead of picking an $r_2$, he picks a pair $(i_2, j_2)$ and so on.

Then, to run the protocol, the entries of the phone book are going to correspond to
\begin{align*}
  P &\in \FF_q[T_1, \dots, T_m, U_1, \dots, U_m] \\
  P(T_1, \dots, T_m, U_1, \dots, U_m) &\coloneqq
    {\wt a}(T_1, \dots, T_m, U_1, \dots, U_m) \wt A(T_1, \dots, T_m) \wt A(U_1, \dots,
    U_m) \\
    &\qquad + \frac{1}{|\HH|^m} {\wt b}(T_1, \dots, T_m) \wt A(T_1, \dots, T_m)
\end{align*}
in place of what we called $P(x,y,z)$ in the sum-check section.

I want to stress now the tilde's above are actually hiding a lot of work.
Let's unpack it a bit: what does $\wt a$ mean?
After all, when you unwind this notational mess we wrote,
we realize that the $a$'s and $b$'s came out of the coefficients of the original
equations $\mathcal E_k$.

The answer is that both Victor and Penny have a lot of arithmetic to do.
Specifically, for Penny,
when she's printing this phone book for $(\lambda_1, \dots, \lambda_E)$,
needs to apply the extension result three times:
\begin{itemize}
  \ii Penny views $a_{\vec\imath, \vec\jmath}$ as a function $\HH^{2m} \to \FF_q$
  and extends it to a polynomial using the above;
  this lets us define
  $\wt a \in \FF_q[T_1, \dots, T_m, U_1, \dots, U_m]$
  as a \emph{bona fide} $2m$-variate polynomial.

  \ii Penny does the same for $\wt b_{\vec\imath}$.

  \ii Finally, Penny does the same on $A \colon \HH^m \to \FF_q$,
  extending it to $\wt A \in \FF_q[T_1, \dots, T_m]$.
  (However, this step is the same across all the phone books, so it only happens once.)
\end{itemize}
Victor has to do the same work for $a_{\vec\imath, \vec\jmath}$ and $b_{\vec\imath}$.
Victor can do this, because he picked the $\lambda$'s,
as he computed the coefficients of his linear combination too.
But Victor does \emph{not} do the last step of computing $\wt A$:
for that, he just refers to the poster Penny gave him,
which conveniently happens to have a table of values of $\wt A$.

Now we can actually finally describe the full contents of the phone book.
It's not simply a table of values of $P$!
We saw in the sum-check protocol that we needed a lot of intermediate steps too
(like the $23T+46$, $161U+23$, $112V+197$).
So the contents of this phone book include, for every index $k$,
every single possible result that Victor would need to run sum-check at the $k$th step.
That is, the $k$th part of this phone book are a big directory where,
for each possible choice of indices $(i_1, \dots, i_{k-1}, j_1, \dots, j_{k-1})$,
Penny has printed the two-variable polynomial in $\FF_q[T,U]$ that arises from sum-check.
(There are two variables rather than one now,
because $(i_k, j_k)$ are selected in pairs.)

This gives Victor a non-interactive way to run sum-check.
Rather than ask Penny, consult the already printed phone book.
Inefficient? Yes. Works? Also yes.

\subsection{Finishing up}
Once Victor runs through the sum-check protocol,
at the end he has a random $(\vec\imath, \vec\jmath)$ and received
the checked the phone book for $P(\vec\imath, \vec\jmath)$.

Assuming it checks out, his other task is to
verify that the accompanying posters that Penny sent ---
that is, the table of values $B_0$ and $B_2$ associated to $\wt A$ ---
look like they mostly come from a low-degree polynomial.
Unlike the sum-check step where we needed to hack the earlier procedure,
this step is a direct application of line-versus-point test, without modification.

Up until now the phone book and posters haven't interacted.
So Victor has to do one more check:
he makes sure that the value of $P(\vec\imath, \vec\jmath)$ he got from the phone book
in fact matches the value corresponding to the poster $B_0$.
In other words, he does the arduous task of computing the extensions
$\wt a$ and $\wt b$, and finally verifies that
\[
  P(\vec\imath, \vec\jmath) \coloneqq
  {\wt a}(\vec\imath, \vec\jmath) B_0[\vec\imath] B_0[\vec\jmath]
  + \frac{1}{|\HH|^m} \wt b(\vec\imath) B_0[\vec\imath]
\]
is actually true.

\subsection{Soundness analysis}
\todo{write this}

\section{Reasons to not be excited by the above protocol}
The previous section describes a long procedure that has a PCP flavor,
but it suffers from several issues (which is why we it's as a toy example).
\begin{itemize}
  \ii \textbf{Amount of reading}:
  The amount of reading on Victor's part is not $O(1)$ like we promised.
  The low-degree testing step with the posters used $O(1)$ entries,
  but the sum-check required reading roughly
  \[ O(|\mathbb H|^2) \cdot (m+O(1)) \approx \frac{(\log N)^3}{\log \log N} \]
  entries from the phone book.
  The PCP theorem promises we can get that down to $O(1)$,
  but that's beyond this post's scope.

  \ii \textbf{Length of proof}:
  The procedure above involved mailing $q^E$ phone books,
  which is what we in the business call either ``unacceptably inefficient''
  or ``fucking terrible'', depending on whether you're in polite company or not.
  The next section will show how to get this down to $qEN$ if $q$ is large enough.

  For context, in this protocol one wants a reasonably small prime $q$
  which is about polynomial in $\log(EN)$.
  After all, Quad-SAT is already an NP-complete problem for $q=2$.
  (In contrast, in other unrelated more modern ecosystems,
  the prime $q$ often instead denotes a fixed large prime $q \approx 2^{256}$.)

  \ii \textbf{Time complexity}:
  Even though Victor doesn't read much,
  Penny and Victor both do quite a bit of computation.
  For example,
  \begin{itemize}
  \ii Victor has to compute $\wt a_{\vec\imath, \vec\jmath}$ for his one phone book.
  \ii Penny needs to do it for \emph{every} phone book.
  \end{itemize}

  \ii One other weird thing about this result is that,
  even though Victor has to read only a small part of Penny's proof,
  he still has to read the entire \emph{problem statement},
  that is, the entire system of equations from the original Quad-SAT.
  This can feel strange because for Quad-SAT,
  the problem statement is of similar length to the satisfying assignment!

  \todo{some comments from Telegram here}
\end{itemize}

\section{Reducing the number of phone books --- error correcting codes}
We saw that we can combine all the $E$ equations from Quad-SAT into a single one
by taking a random linear combination.
Our goal is to improve this by taking a ``random-looking'' combination
that still has the same property an assignment failing even one of the $E$ equations
is going to fail the collated equation with probability close to $1$.

It turns out there is actually a well-developed theory of how to take
the ``random-looking'' linear combination,
and it comes from the study of \emph{error-correcting codes}.
We will use this to show that if $q \ge 4(\log (EN))^2$ is large enough,
one can do this with only $q \cdot EN$ combinations.
That's much better than the $q^E$ we had before.

\subsection{A hint}
So uh.
\[ k^1 \mathcal E_1 + k^2 \mathcal E_2 + \dots + k^E \mathcal E_E
  \qquad k = 1, 2, \dots, 100E. \]

\subsection{Definition of error-correcting codes}
An \alert{error-correcting code} is a bunch of codewords
with the property that any two differ in ``many'' places.
An example is the following set of sixteen bit-strings of length $7$:
\begin{align*}
  C = \big\{
    & 0000000, \; 1101000, \; 0110100, \; 0011010, \\
    & 0001101, \; 1000110, \; 0100011, \; 1010001, \\
    & 0010111, \; 1001011, \; 1100101, \; 1110010 \\
    & 0111001, \; 1011100, \; 0101110, \; 1111111 \big\} \subseteq \FF_2^7
\end{align*}
which has the nice property that any two of the codewords in it differ in at least $3$ bits.
This particular $C$ also enjoys the nice property that it's actually
a vector subspace of $\FF_2^7$ (i.e.\ it is closed under addition).
In practice, all the examples we consider will be subspaces,
and we call them \alert{linear error-correcting codes} to reflect this.

When designing an error-correcting code, broadly your goal is to make sure the
minimum distance of the code is as large as possible,
while still trying to squeeze in as many codewords as possible.
The notations used for this
\begin{itemize}
  \ii Usually we let $q$ denote the alphabet size and $n$ the block length
  (the length of the codewords),
  so the codewords live in the set of $q^n$ possible length $n$ strings.

  \ii The \alert{relative distance} is defined as the minimum Hamming distance divided by $n$;
  Higher relative distance is better (more error corrections).

  \ii The \alert{rate} is the $\log_{q^n}(\text{num codewords})$.
  Higher rates are better (more densely packed codewords).
\end{itemize}
So the example $C$ has relative distance $\frac 37$,
and rate $\log_{2^7}(16) = \frac 47$.

\subsection{Examples of error-correcting codes}
\todo{Hadamard, \dots}

\subsection{Composition}
\todo{define, put in example from paper}

\subsection{Recipe}
Compose the Reed-Solomon codes
\begin{align*}
  C_1 &= \mathsf{RS}_{d=E,q=EN} \\
  C_2 &= \mathsf{RS}_{d=\log(EN),q}.
\end{align*}
This gives a linear code corresponding to an $s \times m$ matrix $M$,
where $s \coloneqq q \cdot EN$, which has relative distance at least $1 - 1/\sqrt q$.
The rows of this matrix (just the rows, not their row span)
then correspond to the desired linear combinations.

%Phone book = content needed
%
%So now, we have this single equation that we want to verify rather than a system.
%The problem is the this equation was obtained by mashing together a really long sequence
%of equations in tons of variables,
%so it's really not fun to verify even if the individual equations
%in the original instance were reasonable.
%
%\subsection{Setup}
%
%(In the context of Quad-SAT, each $Z_i$ is either a multiple of $x_i x_j$ or $x_i$,
%but this protocol works more broadly for any sort of long addition
%no matter what context the $Z_i$ arise from.)

\appendix

\chapter{PLONK stuff (March 14 office hours, rough live notes)}
\begin{itemize}
  \ii A form of Quad-SAT proving as well, but with shorter proof length
  \ii Yan: zero-knowledge and non-interactivity can be added on afterwards
  usually for cheap
  (hash functions e.g.\ for non-interactivity).
  \ii Needs commitment schemes
\end{itemize}

\section{Choice of NP-complete problem: PLONK}
The ``PLONK equation'' is the standard form
\[ \square a_i + \square b_i + \square c_i + \square a_i b_i = \square \]
where the $\square$ are constants.
We also have a bunch of equalities like $c_7 = a_8$ called copy constraints.
Should be able to massage any Quad-SAT into this form.

So an instance of PLONK problem consists of several PLONK equations
together with some equivalence classes of variables that should be equal.

\section{Polynomial commitment}
Assume $q \equiv 1 \pmod{2^n}$
and let $\omega$ be a $(2^n)$\ts{th} root of unity.
Suppose we have a table that we view as a function
\[ f \colon \{0,1,\dots,2^n-1\} \to \FF_q. \]
Our goal is to make a commitment scheme that allows us to unblind one value.

To do this, we construct a polynomial $P$ of degree $2^n$ satisfying
\[ P(\omega^i) = f(i). \]
There's a primitive that lets us commit $P$.
To be described later.

\section{Back to PLONK instance}
Given PLONK instance, we can assume the number of equations is $2^n$.
Denote the $i$th equation
\[ Q_{L_i} a_i + Q_{R_i} b_i + Q_{O_i} c_i + Q_{M_1} a_i b_i + Q_{c_i}. \]
For each of the eight things, interpolate a polynomial:
e.g.\ for $a_i$, create a polynomial $A$ such that $a_i = A(\omega^i)$.
Then to check all the equations at once,
we plug in random values that aren't powers of $\omega$.

\section{Permutation check}
Copy constraints are a permutation trick.
\begin{align*}
  X(k) &= (\lambda - A(k)) X(k-1) \\
  X'(k) &= (\lambda - A'(k)) X'(k-1)
\end{align*}
Use ordered pairs instead: $(i,a_i) \mapsto i \beta + a_i$.

\chapter{Fully holomorphic encryption office hours}
\section{What is FHE?}
Fully holomorphic encryption (FHE) is a method of encryption
$\opname{enc}$ and $\opname{dec}$ such that for a general function $f$ such that
\[ \opname{dec}(f(\opname{enc}(x))) = f(x). \]
The idea is that if you can use this to outsource a calculation without decrypting data.
For example, if I have this machine learning model $f$ that analyzes some data,
I can give it $\opname{enc}(x)$ to an external cloud provider.

This is already starting to approach ``practical''.
Right now the overhead seems like it will turn out to be 1000x or 10000x
(which is still a lot but actually better than current-state ZK-SNARK).

\section{FHE easily gives multi party computation}
If you have fully holomorphic encryption (FHE) gives 2-party computation with 1 round.

\section{Functional encryption}
\url{https://en.wikipedia.org/wiki/Functional_encryption}

Generate an encrypt key $E$.
Given a function $f$, and a ``decrypt key'' $D_f$ (also depending on $E$)
and methods $\opname{enc}$ and $\opname{dec}$ such that if
\[ c \coloneqq \opname{enc}(x, E) \]
then
\[ \opname{dec}(c, D_f) = f(x). \]
So Alice can send $c$ to Bob and Bob can recover $f(x)$ from it.

This initially seems like a convoluted way of sending $y \coloneqq f(x)$.
Why not just send $y \coloneqq f(x)$ directly using a normal public-key scheme?
Well, the additional benefit here is that you implicitly have a proof you know
a pre-image of $y$, i.e., it's almost like a ZK proof that $f\inv(y)$ exists.

\section{Witness encryption}
Idea: encrypt a secret with a Boolean-output program rather than a key.
The ``password'' is any accepting input to this program.

Example: program is ``correct proof of Riemann hypothesis''.
One can then encrypt a Bitcoin address to claim the Millennium Prize (or other message),
such that whoever first proves the Riemann hypothesis can retrieve this message.

\section{Obfuscation}
In theory, perfect obfuscation would be a black box $f$ such that you can feed
an input into it (un-interactively) and get $f(x)$ without learning about $f$.
This can be proven impossible in theory but some approximations are possible.

This can be thought of as ``functional witness'' encryption.
It generalizes both functional encryption and witness encryption.

\end{document}
